```latex
\documentclass{article}
\usepackage{amsmath}
\usepackage{graphicx}
\usepackage{hyperref}

\title{Efficient Knowledge Integration for Scalable and Generalizable Medical AI Agents}
\author{}
\date{}

\begin{document}

\maketitle

\begin{abstract}
Current medical AI agents struggle with the complexity and diversity of medical knowledge. This proposal outlines research to develop more efficient knowledge integration methods for improved scalability and generalizability. We aim to surpass the limitations of existing approaches like MedResearcher-R1 by exploring novel architectures and training strategies. Our focus is on creating resource-efficient agents capable of handling diverse medical subspecialties, ultimately leading to more reliable and accessible medical AI solutions.
\end{abstract}

\section{Introduction}
Medical AI agents hold immense potential for question answering and knowledge discovery, but often lack deep domain understanding. While approaches like MedResearcher-R1 leverage knowledge graphs, their scalability and generalizability remain unclear \cite{medresearcher}. This research addresses these limitations by investigating more efficient knowledge integration methods. Our objectives are to develop a novel architecture, evaluate its performance across diverse medical subspecialties, and improve resource efficiency for broader applicability.

\section{Methodology}
We propose a hybrid architecture combining a transformer-based language model with a lightweight knowledge graph embedding module. This module will be trained using contrastive learning to efficiently encode medical concepts and relationships. We will evaluate performance on a diverse set of medical QA datasets, covering various subspecialties. Experimental design includes comparing our approach against MedResearcher-R1 and other state-of-the-art methods using metrics like accuracy, F1-score, and inference time.

\section{Expected Results}
We anticipate achieving significant improvements in both accuracy and efficiency compared to existing methods. We expect our architecture to demonstrate superior performance across diverse medical subspecialties, indicating improved generalizability. We will quantify these improvements using standard metrics such as accuracy, F1-score, and inference time, comparing our results against baselines and MedResearcher-R1.

\section{Discussion}
This research has the potential to significantly advance the field of medical AI by enabling the development of more reliable and accessible agents. Limitations include the reliance on existing knowledge graphs and the potential for bias in the training data. Future work will focus on incorporating private medical retrieval engines and exploring methods for mitigating bias.

\section{Conclusion}
This research proposes a novel approach to knowledge integration for medical AI agents, aiming to improve scalability, generalizability, and resource efficiency. The expected outcomes will contribute to the development of more effective and reliable medical AI solutions.

\begin{thebibliography}{9}
\bibitem{medresearcher} Placeholder for MedResearcher-R1 citation.
\end{thebibliography}

\end{document}
```