```latex
\documentclass{article}
\usepackage{amsmath}
\usepackage{graphicx}
\usepackage{hyperref}

\title{Prompt Robustness Score: Quantifying and Enhancing Transferability in Prompt Engineering}
\author{}
\date{}

\begin{document}

\maketitle

\begin{abstract}
Current prompt engineering research lacks a systematic evaluation of prompt transferability across diverse models and datasets. This paper introduces the "Prompt Robustness Score" (PRS), a novel metric to quantify prompt performance consistency. We aim to identify prompt design principles that maximize transferability, reducing the need for model-specific optimization. Our methodology involves evaluating prompts across various models and datasets, analyzing PRS variations, and developing guidelines for robust prompt construction. Expected outcomes include a validated PRS metric and actionable insights for creating transferable prompts, advancing the field of prompt engineering.
\end{abstract}

\section{Introduction}
Prompt engineering has emerged as a crucial technique for harnessing the power of large language models (LLMs). However, prompts often exhibit brittle performance, failing to generalize across different model architectures and datasets. While techniques like prompt tuning exist, their transferability remains poorly understood. This paper addresses this gap by introducing a novel metric, the Prompt Robustness Score (PRS), to quantify prompt transferability. Our objectives are to define PRS, identify prompt design principles that maximize transferability, and minimize model-specific prompt optimization.

\section{Methodology}
Our approach involves three key stages: (1) \textbf{Prompt Evaluation}: We will evaluate a diverse set of prompts across multiple LLMs (e.g., GPT-3, LLaMA) and datasets (e.g., text summarization, question answering). (2) \textbf{PRS Calculation}: We will define PRS based on performance variance across models and datasets. Specifically, PRS will be calculated as the inverse of the standard deviation of performance metrics (e.g., ROUGE, BLEU) across different settings. (3) \textbf{Design Principle Identification}: We will analyze the relationship between prompt characteristics (e.g., length, complexity, use of keywords) and PRS to identify design principles for robust prompts. We will use statistical analysis and machine learning techniques to uncover these relationships.

\section{Expected Results}
We expect to develop a validated PRS metric that accurately reflects prompt transferability. We anticipate identifying specific prompt design principles (e.g., using abstract language, avoiding model-specific jargon) that significantly improve PRS. We will compare the performance of prompts designed using these principles against baseline prompts optimized for individual models. We will measure performance using standard NLP metrics like ROUGE, BLEU, and accuracy, and demonstrate a statistically significant improvement in PRS for prompts designed using our identified principles.

\section{Discussion}
This research has significant implications for the practical application of prompt engineering, enabling the creation of more adaptable and reusable prompts. Limitations include the computational cost of evaluating prompts across numerous models and datasets. Future work will focus on developing automated methods for predicting PRS and exploring the application of PRS to other areas of prompt engineering, such as few-shot learning.

\section{Conclusion}
This paper proposes a novel approach to quantifying and enhancing prompt transferability through the introduction of the Prompt Robustness Score. By identifying key design principles, we aim to advance the field of prompt engineering and enable the creation of more robust and adaptable prompts.

\bibliographystyle{plain}
\begin{thebibliography}{9}
\bibitem{} Virtual Community paper (Hypothetical)
\bibitem{} MedResearcher-R1 paper (Hypothetical)
\end{thebibliography}

\end{document}
```