\documentclass{article}
\usepackage{amsmath}
\usepackage{graphicx}
\usepackage{cite}

\title{Prompt Engineering for Enhanced Reasoning in Large Language Models: A Knowledge Graph Augmented Approach}

\begin{document}

\maketitle

\begin{abstract}
Large Language Models (LLMs) exhibit impressive capabilities, but their reasoning abilities often falter in complex scenarios. This proposal investigates a novel prompt engineering technique leveraging knowledge graph augmentation to improve LLM reasoning. We aim to construct prompts that explicitly incorporate relevant knowledge graph information, guiding the LLM towards more accurate and reliable inferences. Our approach will be evaluated on benchmark reasoning datasets, demonstrating the efficacy of knowledge graph augmented prompts in enhancing LLM performance and mitigating common reasoning errors.
\end{abstract}

\section{Introduction}
LLMs have shown remarkable progress, yet their reasoning remains a challenge. Existing prompt engineering techniques often rely on handcrafted templates or automated search, lacking explicit knowledge integration. This research addresses this gap by exploring knowledge graph augmented prompts. Our objective is to develop a systematic approach for incorporating relevant knowledge graph information into prompts, enabling LLMs to perform more accurate and robust reasoning. We will compare our approach with standard prompt engineering methods and fine-tuning strategies.
\section{Methodology}
Our approach involves three key steps: (1) Knowledge Graph Selection: Choosing appropriate knowledge graphs relevant to the reasoning task. (2) Prompt Construction: Developing algorithms to extract relevant entities and relationships from the knowledge graph and incorporate them into the prompt. (3) LLM Inference: Feeding the augmented prompts to the LLM and evaluating its performance. We will use benchmark reasoning datasets like CommonsenseQA and StrategyQA. Performance will be measured using accuracy, precision, and recall.
\section{Expected Results}
We expect that knowledge graph augmented prompts will significantly improve LLM reasoning performance compared to standard prompts. We anticipate higher accuracy and reduced hallucination rates. We will compare our approach against baseline prompt engineering techniques and fine-tuned LLMs. We will also analyze the impact of different knowledge graph selection strategies on the overall performance.
\section{Discussion}
This research has the potential to significantly enhance LLM reasoning capabilities. Limitations include the computational cost of knowledge graph processing and the potential for bias in the knowledge graph. Future work will explore methods for automatically selecting and integrating knowledge graphs, as well as mitigating bias in the augmented prompts.
\section{Conclusion}
This proposal outlines a novel approach to prompt engineering that leverages knowledge graph augmentation to improve LLM reasoning. Our research aims to develop a systematic and effective method for incorporating external knowledge into prompts, leading to more accurate and reliable LLM inferences.
\bibliographystyle{plain}
\begin{thebibliography}{9}
\end{thebibliography}

\end{document}