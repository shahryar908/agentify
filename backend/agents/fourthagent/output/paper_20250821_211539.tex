```latex
\documentclass{article}
\usepackage{amsmath}
\usepackage{graphicx}
\usepackage{hyperref}

\title{Prompt Engineering for Adaptive Medical Reasoning: Bridging the Gap Between LLMs and Clinical Practice}
\author{}
\date{}

\begin{document}

\maketitle

\begin{abstract}
Large Language Models (LLMs) show promise in medical reasoning, but struggle with evolving knowledge and real-world clinical workflows. This paper proposes a novel prompt engineering framework, \textit{Adaptive Clinical Prompting (ACP)}, that dynamically integrates knowledge graph updates and simulates clinical decision-making scenarios. ACP leverages a meta-prompting strategy to adapt to new medical information and patient contexts. We hypothesize that ACP will improve diagnostic accuracy and treatment planning compared to standard LLM prompting techniques.
\end{abstract}

\section{Introduction}
LLMs face challenges in specialized domains like medicine due to their limited ability to handle complex, evolving knowledge. Existing approaches using knowledge graphs and specialized retrieval engines improve performance, but adaptability to new information and integration with clinical workflows remain limited. This paper addresses this gap by developing Adaptive Clinical Prompting (ACP), a novel prompt engineering framework that enables dynamic knowledge integration and clinical simulation, thereby improving diagnostic accuracy and facilitating better treatment planning. Our objective is to create a more robust and adaptable medical reasoning system.

\section{Methodology}
ACP employs a meta-prompting strategy. First, a knowledge graph is continuously updated with new medical literature. Second, a meta-prompt is designed to guide the LLM in retrieving relevant knowledge from the graph based on patient symptoms and medical history. Third, the LLM simulates clinical decision-making by generating differential diagnoses and treatment plans. We will evaluate ACP using a dataset of clinical case studies, comparing its performance against standard prompting techniques and existing knowledge-informed LLMs. Metrics include diagnostic accuracy, treatment plan appropriateness, and response time.

\section{Expected Results}
We expect ACP to outperform standard prompting techniques in diagnostic accuracy and treatment plan appropriateness. We hypothesize that the dynamic knowledge integration and clinical simulation capabilities of ACP will lead to a significant improvement in these metrics. We will compare ACP's performance against baseline LLMs and existing knowledge-informed models, demonstrating its superior adaptability and clinical reasoning capabilities. We anticipate a 15\% improvement in diagnostic accuracy compared to standard prompting.

\section{Discussion}
The successful implementation of ACP has significant implications for improving medical decision-making and patient care. However, limitations include the reliance on the quality of the knowledge graph and the computational cost of dynamic knowledge retrieval. Future work will focus on optimizing the knowledge graph update process and exploring more efficient retrieval mechanisms. We will also investigate the ethical implications of using LLMs in clinical settings.

\section{Conclusion}
This paper proposes Adaptive Clinical Prompting (ACP), a novel prompt engineering framework designed to bridge the gap between general LLM capabilities and the nuanced demands of medical reasoning. ACP's dynamic knowledge integration and clinical simulation capabilities promise to improve diagnostic accuracy and treatment planning, ultimately leading to better patient outcomes.

\begin{thebibliography}{9}
\bibitem{paper1} (Hypothetical Paper 1 Citation)
\bibitem{paper2} (Hypothetical Paper 2 Citation)
\end{thebibliography}

\end{document}
```