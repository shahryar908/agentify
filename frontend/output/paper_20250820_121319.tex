\documentclass{article}
\usepackage[utf8]{inputenc}
\usepackage{amsmath, amsfonts, amssymb}
\usepackage{graphicx}
\usepackage[margin=1in]{geometry}

\begin{document}

Of course. Here is a research paper title and abstract formatted in LaTeX, focusing on addressing common gaps in few-shot prompt engineering, such as the ad-hoc nature of example selection and the lack of a systematic, adaptive framework.

```latex
\title{Prompt Forging: A Principled Framework for Automated Few-Shot Exemplar Selection and Structuring}
\author{Research Team}
\date{\today}

\begin{abstract}
Few-shot prompting for Large Language Models (LLMs) is a powerful paradigm, yet its effectiveness is critically dependent on the manual, ad-hoc selection of in-context examples. This process is often brittle, suboptimal, and lacks a systematic foundation, creating a significant barrier to reliable performance. We introduce Prompt Forging, a novel framework that automates and optimizes few-shot prompt construction. Our approach treats exemplar selection as a principled retrieval problem, dynamically sourcing and structuring a diverse set of examples from a large candidate pool based on the input query's semantic features. Key contributions include: (1) the Prompt Forging algorithm for query-adaptive exemplar selection; (2) empirical validation showing significant performance gains over static and random selection methods across complex reasoning and code generation benchmarks; and (3) a formalization of the prompt optimization space. This work transitions prompt engineering from an art to a reproducible science, enhancing the efficiency and robustness of LLMs in low-data scenarios.
\end{abstract}
```

\maketitle





Of course. Based on the provided context of "few-shot prompt engineering," here are realistic and focused "Expected Results" and "Conclusion" sections.

***

### **Expected Results**

We expect our proposed method for automated few-shot example selection to demonstrate a marked improvement over baseline prompting strategies across a variety of natural language understanding tasks. Specifically, we anticipate the following outcomes:

1.  **Performance Improvement on Standard Benchmarks:** When evaluated on tasks such as sentiment analysis (e.g., SST-2), natural language inference (e.g., MNLI), and question-answering (e.g., SQuAD 2.0), we hypothesize that our method will yield a 5-10% absolute improvement in accuracy compared to a standard few-shot approach using randomly selected examples. This improvement is expected to be most pronounced in complex or nuanced tasks where the choice of examples is critical for establishing the desired inference pattern.

2.  **Increased Robustness and Lower Performance Variance:** A key expected result is a significant reduction in performance variance across different runs. While random example selection can lead to high variability depending on the quality of the chosen "shots," our systematic approach is expected to produce consistently high-performing prompts. We will demonstrate this by showing a lower standard deviation in accuracy over multiple trials compared to the baseline.

3.  **Superior Performance to Heuristic-Based Selection:** We will compare our method against existing heuristic-based selection strategies (e.g., selecting examples based on semantic similarity to the input query). We expect our model-driven optimization approach to outperform these heuristics, as it can capture more subtle and complex relationships between examples that are conducive to better in-context learning.

The significance of these results lies in providing a more reliable and effective method for leveraging the in-context learning capabilities of large language models. By automating and optimizing a critical step in prompt engineering, our work aims to make few-shot prompting a more dependable tool for practitioners, reducing the need for manual trial-and-error and unlocking higher performance ceilings.

### **Conclusion**

This paper introduced a novel framework for the automated selection and structuring of few-shot examples in prompt engineering. Our primary contribution is an algorithmic approach that moves beyond random or simple heuristic-based selection, instead optimizing for examples that collectively maximize a model's in-context learning ability for a given task.

The anticipated results suggest that our method not only enhances task performance on several key benchmarks but also significantly increases the reliability and stability of few-shot prompting. The key implication of this work is that the process of prompt engineering can be made more systematic and less of an art. By formalizing example selection, we lower the barrier for developers to build robust and high-performing LLM-powered applications, making state-of-the-art results more accessible and reproducible.

Despite these promising outcomes, our work has limitations. The computational overhead of our selection algorithm is higher than that of random sampling, and its effectiveness has primarily been validated on English-language text-based tasks. Future work will proceed in three main directions. First, we will explore methods to optimize the efficiency of the selection algorithm to make it viable for real-time applications. Second, we plan to extend the framework to multilingual and multi-modal contexts, such as vision-language models. Finally, we aim to investigate the theoretical underpinnings of why certain example sets are more effective than others, which could lead to more generalizable principles for in-context learning.

\end{document}