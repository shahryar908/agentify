```latex
\documentclass{article}
\usepackage{amsmath}
\usepackage{amssymb}
\usepackage{graphicx}
\usepackage{cite}

\title{Sentiment-Driven Prompt Engineering for Enhanced Robot Learning in Healthcare}
\author{}
\date{}

\begin{document}

\maketitle

\begin{abstract}
This paper proposes a novel approach to enhance robot learning in healthcare settings by leveraging sentiment analysis of user feedback. We aim to bridge the gap between natural language processing and robotics by developing prompt engineering techniques that enable Large Language Models (LLMs) to interpret user sentiment regarding robot assistance. This information will then be translated into actionable insights for improving robot behavior. Our research will contribute to more intuitive and user-friendly robotic healthcare solutions.
\end{abstract}

\section{Introduction}
Robotic assistance in healthcare holds immense potential, but user acceptance hinges on intuitive interaction. Existing research demonstrates LLMs' capabilities in sentiment analysis and imitation learning for robotics. However, a gap exists in combining these approaches to improve robot behavior based on user feedback. This paper addresses this gap by exploring prompt engineering techniques to guide LLMs in interpreting user sentiment regarding robot assistance and generating actionable insights for robot learning. Our objective is to develop a framework that translates user sentiment into improved robot performance.

\section{Methodology}
Our approach involves collecting user feedback on robot interactions in a simulated healthcare environment. We will then employ prompt engineering techniques to guide an LLM in analyzing the sentiment expressed in the feedback. Specifically, we will design prompts that elicit actionable insights for robot behavior modification. The LLM's output will be used to adjust robot control parameters using reinforcement learning. We will evaluate the system's performance by measuring improvements in user satisfaction and task completion rates.

\section{Expected Results}
We expect to demonstrate that sentiment-driven prompt engineering can significantly improve robot learning and user satisfaction. We will measure the effectiveness of our approach using metrics such as user satisfaction scores, task completion rates, and the correlation between sentiment analysis and robot behavior adjustments. We will compare our approach to baseline methods that do not incorporate sentiment analysis. We anticipate a significant improvement in robot performance and user experience.

\section{Discussion}
This research has the potential to improve the usability and acceptance of robots in healthcare. Limitations include the reliance on simulated environments and the potential for bias in the LLM's sentiment analysis. Future work will focus on deploying the system in real-world healthcare settings and exploring methods to mitigate bias in sentiment analysis. We will also investigate the use of more sophisticated prompt engineering techniques.

\section{Conclusion}
This paper proposes a novel approach to enhance robot learning in healthcare by leveraging sentiment analysis of user feedback. Our research will contribute to more intuitive and user-friendly robotic healthcare solutions.

\bibliographystyle{unsrt}
\begin{thebibliography}{9}
\bibitem{paper1} Placeholder for Paper 1 citation.
\bibitem{paper2} Placeholder for Paper 2 citation.
\end{thebibliography}

\end{document}
```